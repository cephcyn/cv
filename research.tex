\begin{rubric}{Research}

\entry*[2022 -- $\cdots\cdot$] % 2022jun - 20XX XXX
	\textbf{GMSE Researcher} with NIST, mentored by \pSextonT{}.
% 	\par [TODO NIST projects]
% \entry*[2021 -- $\cdots\cdot$] % 2021aug - 20XX XXX
% 	\textbf{Graduate Researcher} with \pPiersonE{}.
% 	\par [TODO MuckRock]
\entry*[2021 -- $\cdots\cdot$] % 2021aug - 20XX XXX
	\textbf{Graduate Researcher} with \pJoachimsT{}.
	\par Exploring the affordances of using large language models within recommendation systems \cite{gaoEndToEnd2024, zhouLanguageBasedUser2024, zhouGPTBaselineRecommendation2023}, the feasibility of explanations as an auditing technique \cite{zhouHowExplainJustify2023, zhou2022how}. Contributions to analyses of holistic review in undergraduate admissions \cite{leeEndingAffirmativeAction2024, leeAugmentingHolisticReview2023}.
\entry*[2020 -- 2021] % 2020apr - 2021aug
	\textbf{Graduate Researcher} with \pGlassmanE{} (Harvard) and \pWeldD{} (UW).
	\par Developed an interactive, human-AI collaborative aggregation and visualization method for sensemaking content in research paper abstracts.
	\par Wrote up methods and design process in Master's thesis \cite{zhou2021thesis}.
\entry*[2019 -- 2021] % 2019aug - 2021aug
	\textbf{Graduate Researcher}, \href{http://hai.cs.washington.edu}{Lab for Human-AI Interaction} (University of Washington)
	\par Mentored by \pBansalG{} and advised by \pWeldD{}.
	\par Developed, implemented, and evaluated a novel adaptive explanation style for human-AI teams on a sentiment analysis task. Analyzed participants' feedback on how AI explanations impacted their decision-making.
	\par Resulted in 2nd/3rd-author CHI publication \cite{bansal2021does}. Also featured in \href{https://sites.google.com/view/whi2020/home}{a WHI 2020 spotlight} \cite{bansal2020does}.
\entry*[2018 -- 2019] % 2018sep – 2019aug
    \textbf{Undergraduate Researcher} with \pRuzzoL{} (University of Washington)
    \par Developed a set of tools (\textit{blockmerge} and \textit{crosscompare}) and a pipeline centered on CMfinder to search for potentially structured fRNA sequences across alignment block boundaries and cluster found covariance models.
    \par Wrote up methods and findings in Bachelor's thesis \cite{zhou2019thesis}.
\entry*[2018 -- 2019] % 2018mar - 2019jun
	\textbf{Undergraduate Researcher} with \pPahnkeE{} (Foster School of Business, UW)
	\par Collected (with partial automation), organized, and cleaned data from a diverse range of websites (social media, blogs, business homepages) to form an original data set.
\end{rubric}